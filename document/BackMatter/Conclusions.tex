\begin{conclusions}

Este trabajo ha demostrado la viabilidad de incorporar parámetros de asimetría direccional en el \textit{Gaussian Splatting}, desarrollando el \textit{Directional Asymmetric Gaussian Splatting (DA-GS)} como una extensión que enriquece significativamente la capacidad expresiva del modelo original. La introducción de cuatro parámetros adicionales ($s_x$, $s_y$, $s_z$, $S$) que controlan la dirección y magnitud de la asimetría mediante la función $A \cdot (1-e^{-S \cdot B})$ ha resultado en una formulación matemáticamente sólida y computacionalmente eficiente que preserva la diferenciabilidad y compatibilidad estructural del algoritmo base.

Los resultados experimentales confirman que las gaussianas con asimetría direccional poseen mayor capacidad para representar bordes y discontinuidades, manifestándose en una eficiencia representacional superior que permite lograr reconstrucciones de calidad comparable utilizando menos primitivas gaussianas. Los vectores de asimetría aprendidos muestran una correspondencia notable con las características geométricas de los objetos reconstruidos, lo que sugiere que las gaussianas asimétricas capturan información intrínseca valiosa para aplicaciones posteriores.

La extensión desarrollada mantiene la eficiencia computacional característica del \textit{Gaussian Splatting} original, con un incremento marginal de complejidad del 7.1\% que resulta en beneficios sustanciales en capacidad expresiva. Aunque las métricas de calidad de reconstrucción no superaron consistentemente al método original, la evidencia experimental revela un potencial considerable que puede materializarse mediante futuras optimizaciones. La eficiencia representacional observada y la capacidad de las gaussianas asimétricas para capturar características geométricas específicas indican que refinamientos adicionales en la simulación de la asimetría pueden conducir a mejoras cuantitativas significativas.

Este trabajo se inscribe en la línea de investigación que busca nuevas primitivas geométricas para mejorar la representación de escenas tridimensionales, contribuyendo al creciente corpus de métodos que extienden las capacidades del \textit{Gaussian Splatting} tradicional. La investigación demuestra que es posible superar las limitaciones fundamentales del método base sin comprometer su eficiencia computacional, estableciendo un precedente metodológico importante. La formulación matemática elegante y la implementación práctica proporcionan herramientas concretas que la comunidad científica puede utilizar, adaptar y perfeccionar. Las bases sólidas establecidas abren múltiples líneas de investigación futura, incluyendo formulaciones matemáticas alternativas, optimización de hiperparámetros del modelo actual y extensión a escenas dinámicas.
Este trabajo ha demostrado la viabilidad de incorporar parámetros de asimetría direccional en el \textit{Gaussian Splatting}, desarrollando el \textit{Directional Asymmetric Gaussian Splatting (DA-GS)} como una extensión que enriquece significativamente la capacidad expresiva del modelo original. La introducción de cuatro parámetros adicionales ($s_x$, $s_y$, $s_z$, $S$) que controlan la dirección y magnitud de la asimetría mediante la función $A \cdot (1-e^{-S \cdot B})$ ha resultado en una formulación matemáticamente sólida y computacionalmente eficiente que preserva la diferenciabilidad y compatibilidad estructural del algoritmo base.

Los resultados experimentales confirman que las gaussianas con asimetría direccional poseen mayor capacidad para representar bordes y discontinuidades, manifestándose en una eficiencia representacional superior que permite lograr reconstrucciones de calidad comparable utilizando menos primitivas gaussianas. Los vectores de asimetría aprendidos muestran una correspondencia notable con las características geométricas de los objetos reconstruidos, lo que sugiere que las gaussianas asimétricas capturan información intrínseca valiosa para aplicaciones posteriores.

La extensión desarrollada mantiene la eficiencia computacional característica del \textit{Gaussian Splatting} original, con un incremento marginal de complejidad del 7.1\% que resulta en beneficios sustanciales en capacidad expresiva. Aunque las métricas de calidad de reconstrucción no superaron consistentemente al método original, la evidencia experimental revela un potencial considerable que puede materializarse mediante futuras optimizaciones. La eficiencia representacional observada y la capacidad de las gaussianas asimétricas para capturar características geométricas específicas indican que refinamientos adicionales en la simulación de la asimetría pueden conducir a mejoras cuantitativas significativas.

Este trabajo se inscribe en la línea de investigación que busca nuevas primitivas geométricas para mejorar la representación de escenas tridimensionales, contribuyendo al creciente corpus de métodos que extienden las capacidades del \textit{Gaussian Splatting} tradicional. La investigación demuestra que es posible superar las limitaciones fundamentales del método base sin comprometer su eficiencia computacional, estableciendo un precedente metodológico importante. La formulación matemática elegante y la implementación práctica proporcionan herramientas concretas que la comunidad científica puede utilizar, adaptar y perfeccionar. Las bases sólidas establecidas abren múltiples líneas de investigación futura, incluyendo formulaciones matemáticas alternativas, optimización de hiperparámetros del modelo actual y extensión a escenas dinámicas.

Los fundamentos aquí establecidos trascienden la contribución técnica inmediata, abriendo nuevos horizontes en la representación geométrica tridimensional.  El \textit{DA-GS} crea un marco metodológico que facilita su adopción y el desarrollo incremental en la comunidad científica, con un potencial considerable para futuras optimizaciones que pueden materializar plenamente sus ventajas teóricas.
Los fundamentos aquí establecidos trascienden la contribución técnica inmediata, abriendo nuevos horizontes en la representación geométrica tridimensional.  El \textit{DA-GS} crea un marco metodológico que facilita su adopción y el desarrollo incremental en la comunidad científica, con un potencial considerable para futuras optimizaciones que pueden materializar plenamente sus ventajas teóricas.
\end{conclusions}