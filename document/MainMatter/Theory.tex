\chapter{Marco Teórico}\label{chapter:theory}

\section{Notación y convenciones empleadas en la tesis}
En esta tesis se adopta una notación matemática consistente para facilitar la lectura. A continuación se resumen las convenciones principales:

\textbf{Vectores:} Se denotan con letra minúscula en negrita, por ejemplo $\mathbf{x}$. Un vector columna tridimensional se escribe como 
$\mathbf{x} = (x, y, z)^\top$, donde el superíndice $\top$ indica la transposición (es decir, $\mathbf{x}^\top$ es un vector fila).

\textbf{Escalares:} Se representan con letra minúscula en estilo normal o itálico, por ejemplo $s \in \mathbb{R}$, $t$ para el tiempo, etc. 
Los parámetros individuales de vectores o matrices también se denotan con letras normales (por ejemplo $x$, $y$, $z$ para las componentes de $\mathbf{x}$).

\textbf{Matrices:} Se denotan con letra mayúscula (habitualmente en negrita o con una letra griega mayúscula). Por ejemplo, $\Sigma$ 
representa una matriz de covarianza $3\times 3$. La transpuesta de una matriz $M$ se indica como $M^\top$.

\textbf{Operadores y derivadas:} $\nabla$ indica el operador gradiente. La notación $\frac{\partial}{\partial x}$ denota derivada parcial 
respecto a $x$. En general, $f_x$ o $\partial f/\partial x$ indican la derivada parcial de $f$ con respecto a $x$. Para derivadas temporales 
se emplea $\frac{d}{dt}$ o $\dot{f}(t)$ según convenga.

\textbf{Funciones Gaussianas:} La distribución normal (Gaussiana) univariante con media $\mu$ y varianza $\sigma^2$ se denota $\mathcal{N}(\mu,\sigma^2)$. 
En el caso multivariado, una distribución Gaussiana en $\mathbb{R}^d$ con media vectorial $\boldsymbol{\mu} \in \mathbb{R}^d$ y matriz de 
covarianza $\Sigma \in \mathbb{R}^{d\times d}$ (simétrica definida positiva) se denota $\mathcal{N}(\boldsymbol{\mu}, \Sigma)$. 
Su función de densidad de probabilidad (FDP) viene dada por:
$$f(\mathbf{x}) = \frac{1}{(2\pi)^{d/2}\,|\Sigma|^{1/2}} \exp\!\Big(-\frac{1}{2}(\mathbf{x}-\boldsymbol{\mu})^\top \,\Sigma^{-1}\,(\mathbf{x}-\boldsymbol{\mu})\Big)\,.$$

\textbf{Coordenadas y sistemas de referencia:} Salvo indicación en contrario, se emplea un sistema de coordenadas 3D derecho. 
Un punto en el espacio 3D se denota $(x, y, z)$ en coordenadas cartesianas. Cuando se utilizan coordenadas homogéneas para transformaciones proyectivas, 
se extiende como $(x, y, z, 1)$. Para las transformaciones de vista (cámara), $M_{wc}$ denota la matriz $4\times 4$ de transformación de mundo a cámara 
(que incluye rotación $R_{wc}$ y traslación). Se asume la convención de cámara pinhole con eje $Z$ apuntando hacia adelante en el espacio de la cámara, 
de modo que la proyección perspectiva estándar de un punto $\mathbf{X} = (X, Y, Z)^\top$ en coordenadas de cámara es 
$(u, v) = \big(f \frac{X}{Z}, f \frac{Y}{Z}\big)$ (suponiendo distancia focal $f$ y coordenadas centradas).

\textbf{Espacio de color:} Los colores se representarán típicamente como vectores en el espacio RGB. En contextos de renderizado físico, 
se utilizará RGB en espacio lineal (no corregido por gamma) y, de ser necesario, se aclarará si se utilizan otros espacios de color (por ejemplo YUV, HSV) 
o si se aplica alguna transformación de color. Un color se denotará como $\mathbf{c} = (R, G, B)$ con componentes en el rango apropiado (p. ej. $[0,1]$ 
para intensidades normalizadas).

Con estas convenciones, las expresiones matemáticas en la tesis mantienen consistencia. Por ejemplo, $\mathbf{v}^\top \mathbf{w}$ representa el 
producto punto de dos vectores $\mathbf{v}$ y $\mathbf{w}$, $| \mathbf{x} |$ la norma euclídea de $\mathbf{x}$, y $\nabla f(\mathbf{x})$ el gradiente 
de $f$ respecto a $\mathbf{x}$. Asimismo, $\mathbb{E}[X]$ denota la esperanza de una variable aleatoria $X$ y $\operatorname{Var}(X)$ su varianza. 
Estas notaciones y símbolos serán utilizados a lo largo del capítulo para describir formalmente los conceptos teóricos.

\section{Espacios Gaussianos y representación en 3D}
En gráficos por computadora y visión 3D, es posible representar objetos tridimensionales mediante distribuciones Gaussianas en el espacio. 
La idea básica es modelar la forma y apariencia de un objeto como una suma (o mezcla) de funciones gaussianas centradas en diferentes ubicaciones 
del espacio 3D. Cada Gaussiana 3D aporta una “mancha” o splat elipsoidal cuyo volumen abarca una parte del objeto. Matemáticamente, una Gaussiana en 
3D se caracteriza por un centroide (media) $\boldsymbol{\mu} = (\mu_x,\mu_y,\mu_z)^\top$, que indica su posición central en el espacio, y una matriz de 
covarianza $\Sigma$, que describe la dispersión de la distribución alrededor del centroide en las tres dimensiones. La matriz $\Sigma$ determina la forma 
y orientación del lóbulo Gaussiano: sus valores propios $\lambda_1,\lambda_2,\lambda_3$ (y vectores propios asociados) definen un elipsoide de densidad 
en el espacio, cuyos semiejes tienen longitudes proporcionales a $\sqrt{\lambda_1}, \sqrt{\lambda_2}, \sqrt{\lambda_3}$ y direcciones dadas por los 
vectores propios. Por ejemplo, si $\Sigma = \sigma^2 I$ es proporcional a la identidad, la distribución es isótropa (esfera simétrica); 
valores propios distintos implican una Gaussiana anisótropa cuyos ejes de extensión difieren en longitud, formando un elipsoide. Asimismo, 
la orientación de ese elipsoide en el espacio está dada por la base de vectores propios de $\Sigma$. En resumen, $\boldsymbol{\mu}$ sitúa la Gaussiana 
en el espacio, mientras que $\Sigma$ determina si es una mancha esférica o elongada en alguna dirección y con qué amplitud de dispersión.

En el contexto geométrico, un conjunto suficiente de Gaussianas puede 
aproximar superficies continuas: por ejemplo, una fina Gaussiana anisótropa orientada con su eje menor perpendicular a una superficie puede actuar 
como un surfel (elemento de superficie) que representa localmente un parche de esa superficie. Trabajos fundacionales en gráficos introdujeron 
precisamente los surfels y técnicas de splatting para renderizar nubes de puntos como pequeñas elipses en lugar de píxeles puntuales, evitando 
huecos y aliasing. En esos métodos, cada punto de una nube se renderiza como un disco o elipse (a menudo modelado con un peso Gaussiano) en lugar 
de un punto infinitesimal, lo que genera una superficie continua al proyectar muchos puntos. Por ejemplo, Pfister et al. (2000) propusieron los surfels 
(surface elements) que son esencialmente puntos con área (discos elipsoidales) para reconstruir superficies de alta calidad. Zwicker et al. (2001) 
extendieron esta idea introduciendo filtrado EWA (elliptical weighted average) para el splatting de superficies, reduciendo el aliasing mediante kernels 
Gaussianos anisótropos. Estos trabajos pioneros demostraron que las Gaussianas (o discos con pesos gaussianos) son primitivas efectivas para representar 
y visualizar geometría compleja de manera robusta.

Una propiedad importante de las Gaussianas es que proporcionan un campo continuo de densidad volumétrica. En lugar de representar una superficie solo 
de forma implícita (como en una malla poligonal con bordes definidos), una colección de Gaussianas define una función continua en $\mathbb{R}^3$ 
(por ejemplo, una función de densidad de ocupación o de radiancia). Esto las hace adecuadas para métodos de renderizado volumétrico continuo, 
similares a los campos de radiancia (Neural Radiance Fields, NeRF), pero con la ventaja de ser una representación explícita y localizable espacialmente. 
Recientes investigaciones han confirmado que las Gaussianas 3D constituyen una representación de escena flexible y expresiva, combinando 
lo mejor de representaciones explícitas e implícitas: por un lado, permiten optimización eficiente con calidad visual comparable al estado del arte continuo, 
y por otro, pueden rasterizarse rápidamente mediante proyección al plano de imagen. En particular, Kerbl et al. (2023) introducen una representación 
de escenas basada en millones de Gaussianas anisótropas 3D, demostrando que son capaces de preservar las propiedades de los campos volumétricos continuos 
y al mismo tiempo evitar cálculos ineficientes en regiones vacías del espacio. En su método, cada Gaussiana almacena no solo una densidad sino también 
color (mediante coeficientes de armonicos esféricos para dependencia direccional) y se optimizan iterativamente parámetros como posición y 
matriz de covarianza anisotrópica para ajustar finamente la forma de cada “gota” al contenido real de la escena. El resultado es una representación 
compacta y precisa de la geometría y apariencia: típicamente del orden de 1 a 5 millones de Gaussianas sin estructura de conectividad, capaz de modelar 
tanto partes difusas como detalles finos de la escena.

\section{Proyección a 2D y rasterización diferenciable}
Para visualizar o renderizar las Gaussianas 3D en una imagen 2D, es necesario proyectarlas mediante las transformaciones de cámara estándar y luego 
simular el proceso de rasterización. Proyección 3D a 2D: Dado el centroide $\boldsymbol{\mu}$ de una Gaussiana en coordenadas del mundo, primero se 
lleva a coordenadas de cámara aplicando la transformación rígida correspondiente (rototraslación). Si $\mathbf{X}=(X,Y,Z,1)^\top$ representa la 
posición homogénea de $\boldsymbol{\mu}$ y $M_{wc}$ es la matriz $4\times4$ mundo-a-cámara, entonces $\mathbf{X}_c = M_{wc}\mathbf{X}$ da la posición 
de la Gaussiana en el espacio de la cámara. Aplicando la proyección perspectiva $\Pi$ (definida por los parámetros intrínsecos de la cámara), 
se obtiene la ubicación del centroide en la imagen: $\mathbf{x} = \Pi(\mathbf{X}_c) = (u,v)$. En un modelo de cámara pinhole simple, $\Pi$ 
realiza $u = f X_c/Z_c, v = f Y_c/Z_c$ (asumiendo origen en el centro de la imagen), donde $(X_c, Y_c, Z_c)$ son las coordenadas de $\mathbf{X}_c$ y $f$ 
es la distancia focal.

Cuando se proyecta una distribución Gaussiana, no solo el centro se proyecta sino también su extensión. Intuitivamente, una Gaussiana 3D aparecerá en la 
imagen como una mancha Gaussiana 2D (posiblemente elíptica) alrededor de $(u,v)$. La forma de esta elipse proyectada depende de la orientación de la 
Gaussiana respecto a la cámara y de la profundidad $Z_c$. Un enfoque formal para derivar la distribución proyectada es considerar la transformación 
lineal local inducida por la proyección en las cercanías de $\boldsymbol{\mu}$. Se puede obtener la covarianza proyectada mediante la propagación de 
incertidumbre: dado el Jacobiano $J \in \mathbb{R}^{2\times3}$ de la función de proyección (evaluado en $\boldsymbol{\mu}$), la matriz de covarianza 
en la imagen $\Sigma_{2D}$ se puede aproximar como $\Sigma_{2D} \approx J \Sigma J^\top$. En otras palabras, se proyecta el elipsoide 3D obteniendo 
un elipsoide 2D. Por ejemplo, si la Gaussiana 3D es esférica (misma desviación en todas direcciones) su proyección seguirá siendo aproximadamente 
circular (más pequeña si está lejos debido al factor $1/Z$); si es muy elongada en profundidad, su proyección puede aparecer comprimida. 
Cabe mencionar que, para una transformación de proyección perspectiva completa (no lineal), este cálculo es una aproximación válida para Gaussianas 
suficientemente pequeñas o para propósitos de derivadas. Algunos trabajos recientes incorporan este cálculo explícito: por ejemplo, Conti et al. (2025) 
describen la proyección de Gaussianas 3D a 2D mediante $\boldsymbol{\mu}_{2D} = \pi(T_{wc}\boldsymbol{\mu})$ y $\Sigma_{2D} = J R_{wc}\Sigma R_{wc}^\top J^\top$, 
donde $R_{wc}$ es la rotación de cámara y $\pi$ la proyección intrínseca, lo que coincide con la formulación mencionada.

Rasterización diferenciable: Una vez proyectadas las Gaussianas al plano de imagen, hay que simular el proceso de rasterización, es decir, 
cómo contribuyen a los píxeles de la imagen. A diferencia de una primitiva geométrica tradicional (como un triángulo) que se rasteriza 
marcando píxeles cubiertos, una Gaussiana se rasteriza esparciendo su contribución de forma continua según su perfil de densidad. 
En la práctica, cada Gaussiana proyectada se convierte en una mancha elíptica 2D (un splat). La intensidad de color que dicha Gaussiana 
aporta a un punto de la imagen $(u,v)$ puede modelarse evaluando la función Gaussiana 2D en esa coordenada. Así, el color final de un píxel 
se obtiene combinando las contribuciones de todas las Gaussianas cuya proyección cubre ese píxel. En términos de fórmula, si para un píxel $p$ 
las Gaussianas $i=1\ldots N$ proyectan pesos $w_i(p)$ (obtenidos al evaluar la forma Gaussiana 2D de cada $i$ en la ubicación de $p$) y tienen 
colores $\mathbf{c}_i$, entonces el color resultante es esencialmente un promedio ponderado $\sum_i w_i(p)\mathbf{c}_i$ (normalizado por 
$\sum_i w_i$ o acompañado de una fórmula de composición por alfa, dependiendo del modelo de transparencia). En métodos volumétricos, esta 
combinación se interpreta usualmente mediante compositing frontal-a-dorsal (alpha blending): las Gaussianas más cercanas ocultan (parcialmente) 
a las lejanas según la ecuación de transmisión exponencial (similar al ray-marching de NeRF). De hecho, muchas técnicas de splatting equiparan 
la combinación de puntos a la ecuación de renderizado volumétrico de rayos con alfa pre-multiplicado. Esto permite manejar correctamente la oclusión 
y la acumulación de color de múltiples Gaussianas a lo largo de cada rayo de visión.

Lo crucial del enfoque moderno es que todo el proceso de rasterización se diseña para ser diferenciable de extremo a extremo. Es decir, 
la contribución $w_i(p)$ de cada Gaussiana a cada píxel $p$ es una función suave (por ejemplo, una Gaussiana 2D evaluada en la distancia del 
píxel al centro $\mathbf{x}_i$), por lo que pequeñas variaciones en los parámetros de la Gaussiana ($\boldsymbol{\mu}_i$, $\Sigma_i$, opacidad, 
color, etc.) producen cambios continuos en los píxeles. No hay decisiones binarias bruscas (como dentro/fuera de un triángulo) sino contribuciones 
continuas. Esto habilita el cálculo de gradientes mediante retropropagación, de manera que algoritmos de optimización pueden ajustar los parámetros 
de las Gaussianas comparando la imagen renderizada con una imagen objetivo.

El método Gaussian Splatting de \cite{kerbl20233d}, sobre el cual se cimienta esta tesis es un referente moderno que combina todos los elementos mencionados: 
utiliza millones de 
Gaussianas 3D anisótropas para representar la escena y un algoritmo de rasterización capaz de proyectarlas en tiempo real respetando la oclusión 
y acumulando color de forma difuminada pero exacta. En concreto, implementan un renderizador tile-based en GPU que ordena los splats Gaussianos 
por profundidad y realiza la composición alpha en el orden correcto (de adelante hacia atrás), asegurando así que el resultado coincida con el modelo 
volumétrico físico. Gracias a la naturaleza diferenciable de los splats Gaussianos (suavizados en píxeles), pueden además calcular gradientes 
eficientemente para optimizar las propiedades de cada Gaussiana. El renderizador realiza el backward pass propagando el error de cada píxel 
hacia los splats que contribuyeron a él, permitiendo refinar posición, covarianza, opacidad o color de las Gaussianas. Esta combinación de 
representación Gaussiana 3D + rasterización diferenciable en 2D ha demostrado ser muy potente: logra velocidades de render cercano a tiempo real 
(30 fps a 1080p) con calidad comparable o superior a técnicas basadas en redes neuronales mucho más lentas. En esencia, la proyección y 
rasterización de Gaussianas habilita un pipeline de renderizado continuo donde cada Gaussiana actúa como una primitva analítica suave. La 
diferenciabilidad garantiza que se pueda ajustar la escena automáticamente comparando las imágenes rasterizadas
con las reales, cerrando así el bucle de renderización inversa o reconstrucción 3D asistida por gradientes.

\section{Propiedades de las distribuciones sesgadas (skew-normal multivariante)}
Hasta ahora hemos considerado distribuciones Gaussianas simétricas. Sin embargo, para aumentar la flexibilidad de modelado, 
especialmente cerca de discontinuidades o bordes, es relevante introducir distribuciones sesgadas o con asimetría. Una de las familias más 
conocidas es la distribución skew-normal (o normal asimétrica) propuesta originalmente por Azzalini (1985) y extendida al caso multivariante 
por Azzalini y Dalla Valle (1996). Esta distribución generaliza la normal permitiendo skewness (sesgo) no nulo mediante parámetros adicionales. 
En su forma más simple (univariada), la skew-normal con parámetros de ubicación $\xi$, escala $\omega$ y forma (sesgo) $\alpha$ tiene densidad:
$$f(x) = \frac{2}{\omega} \,\phi\!\Big(\frac{x-\xi}{\omega}\Big)\,\Phi\!\Big(\alpha \,\frac{x-\xi}{\omega}\Big)\,,$$
donde $\phi(\cdot)$ es la función de densidad de una normal estándar (media 0, varianza 1) y $\Phi(\cdot)$ es su función de distribución acumulada. 
Intuitivamente, este formato multiplica la densidad simétrica $\phi$ por un factor $2\,\Phi(\alpha\cdot)$ que inclina el peso hacia un lado u otro 
dependiendo del signo de $\alpha$. Si $\alpha=0$, $\Phi(0)=0.5$ constante y la fórmula se reduce a $\phi$, recuperando la distribución normal clásica 
(sin sesgo). Para $\alpha > 0$, la $\Phi$ introducida acumula más probabilidad en la cola derecha, generando asimetría hacia la derecha; 
si $\alpha < 0$, ocurre lo opuesto (sesgo a la izquierda). En efecto, $\alpha$ controla la asimetría: valores mayores de $|\alpha|$ implican 
distribuciones más sesgadas (con colas más largas hacia un lado). No obstante, la skew-normal mantiene colas "delgadas" (exponenciales) 
similares a la normal, por lo que la asimetría que introduce es moderada; de hecho, se puede demostrar que el coeficiente de sesgo (skewness) 
de esta distribución está acotado en valor absoluto por 1 (en términos estandarizados). La curtosis (apuntamiento) también varía con $\alpha$, 
pero de forma limitada – la familia skew-normal no produce colas pesadas como una t-Student, sino que sigue siendo esencialmente de colas finitas, 
con una curtosis ligeramente distinta a 3 (la de la normal) dependiendo de $\alpha$. Por ejemplo, el exceso de curtosis para una univariante 
viene dado por una fórmula en función de $\delta = \alpha/\sqrt{1+\alpha^2}$, indicando desviaciones moderadas del valor normal.

En el caso multivariante, la distribución skew-normal (SN) se define para un vector aleatorio $\mathbf{X}\in \mathbb{R}^d$ con media 
$\boldsymbol{\xi}\in\mathbb{R}^d$, matriz de dispersión (análogo a covarianza) $\Omega \in \mathbb{R}^{d\times d}$ y vector de forma 
$\boldsymbol{\alpha}\in\mathbb{R}^d$. Una forma posible de su densidad es:
$$f(\mathbf{x}) = 2 \;\phi_d(\mathbf{x}-\boldsymbol{\xi};\,\Omega)\;\Phi\!\big(\boldsymbol{\alpha}^\top \Omega^{-1/2}(\mathbf{x}-\boldsymbol{\xi})\big)\,,$$
donde $\phi_d(\cdot;\Omega)$ es la densidad de una normal $d$-dimensional (media cero, covarianza $\Omega$) y $\Phi(\cdot)$ ahora representa 
la CDF de una normal univariante evaluada en el escalar $\boldsymbol{\alpha}^\top \Omega^{-1/2}(\mathbf{x}-\boldsymbol{\xi})$ 
(que combina las coordenadas de $\mathbf{x}$). Esta construcción asegura que si $\boldsymbol{\alpha}=\mathbf{0}$ se recupera la distribución 
normal $N_d(\boldsymbol{\xi},\Omega)$. La interpretación geométrica de $\boldsymbol{\alpha}$ es menos directa que en 1D, pero esencialmente 
define una dirección en el espacio $\mathbb{R}^3$ hacia la cual la distribución está “sesgada”. Por ejemplo, si $\boldsymbol{\alpha}$ apunta 
en dirección del eje $X$, la distribución tendrá una cola más larga en ese sentido positivo que en el negativo. Una propiedad muy útil de la 
familia skew-normal multivariante (tal como fue concebida por Azzalini y otros) es su cerradura bajo transformaciones lineales. 
Esto significa que si $\mathbf{X}$ sigue una skew-normal multivariante, entonces cualquier transformación lineal $\mathbf{Y}=A\mathbf{X}+\mathbf{b}$ 
produce otra variable $\mathbf{Y}$ que también sigue una distribución skew-normal (con parámetros adecuados transformados). Esta propiedad, 
heredada de la normal, es importante para garantizar consistencia cuando se pasan distribuciones skew-normal a través de matrices de rotación, 
escalado u otros cambios de base – la forma funcional se preserva.

Desde el punto de vista de estadística multivariada, la distribución skew-normal aporta grados de libertad adicionales para ajustar asimetría 
y curtosis de los datos. En contextos donde los datos presentan colas o sesgos (por ejemplo, iluminaciones no balanceadas, 
distribuciones de colores con fondo oscuro y gradientes de sombras, etc.), las gaussianas puramente simétricas pueden ser insuficientes. 
En nuestro caso de representación de geometría 3D, introducir Gaussianas asimétricas podría permitir modelar transiciones bruscas – por ejemplo, 
el borde de un objeto donde a un lado hay superficie (densidad alta) y al otro lado hay vacío (densidad cero) no se modela bien con 
una Gaussiana simétrica, pero una skew-normal podría aproximar mejor esa discontinuidad asignando una cola más larga hacia el interior 
del objeto y un decaimiento rápido hacia el exterior. De forma análoga, en visión por computador se ha demostrado la ventaja de modelos 
de mezcla asimétricos: \cite{2021_Chen} utilizan mezclas de skew-normals para segmentación de imágenes, argumentando que muchas 
distribuciones de intensidad en imágenes reales son asimétricas (por ejemplo, en los bordes de objetos, la distribución de color a uno y 
otro lado difiere) y mostrando que un GMM tradicional (simétrico) falla en capturar esos detalles. Su modelo de mezcla normal asimétrica logra 
resultados más precisos, preservando mejor contornos, esquinas y estructuras delgadas en presencia de ruido. Esto sugiere que, al aplicar 
distribuciones skew-normal en gráficos 3D, podríamos obtener una mayor fidelidad en regiones de bordes o discontinuidades. 
