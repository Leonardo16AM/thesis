\begin{resumen}
	Este estudio tiene como objetivo demostrar que la aplicación tradicional del Gaussian Splatting se puede mejorar mediante la introducción de parámetros adicionales para lograr nuevas formas, mejorando así los resultados
	de la reconstrucción. Específicamente, se investiga el impacto de incorporar la
	asimetría (skewness), que permite a las gaussianas adoptar una gama más amplia de formas. Se realizaron
	evaluaciones experimentales sobre manchas gaussianas 2D para evaluar la efectividad de este enfoque. Los
	resultados indican que las gaussianas con asimetría proporcionan un rendimiento superior en varias métricas. 
	Además, se ha desarrollado un método novedoso para simular la asimetría sin requerir cambios en las técnicas de rasterización existentes. Este enfoque garantiza la compatibilidad
	y la facilidad de integración con los sistemas actuales, allanando el camino para su aplicación en el Gaussian
	Splatting 3D en trabajos futuros. Los hallazgos de esta investigación sugieren que la adición del parámetro de
	asimetría es una mejora valiosa del Gaussian Splatting, y ofrece un potencial significativo para mejorar los
	resultados de la reconstrucción en contextos 2D y 3D.
\end{resumen}

\begin{abstract}
	This study aims to demonstrate that the traditional application of Gaussian splatting can be enhanced by
	introducing additional parameters to achieve new shapes, thereby improving reconstruction results. Specifically,
	this research investigates the impact of incorporating the skewness parameter, which enables Gaussians to adopt
	a wider range of shapes. Experimental evaluations were conducted on 2D Gaussian splatting to assess the
	effectiveness of this approach. The results indicate that Gaussians with skewness provide superior performance
	across various metrics. Moreover, a novel method to simulate skewness
	without requiring changes to existing rasterization techniques has been developed. This approach ensures
	compatibility and ease of integration with current systems, paving the way for its application in 3D Gaussian
	splatting in future work. The findings of this research suggest that the addition of the skewness parameter
	is a valuable enhancement to Gaussian splatting, offering significant potential for improved reconstruction
	outcomes in both 2D and 3D contexts.
\end{abstract}